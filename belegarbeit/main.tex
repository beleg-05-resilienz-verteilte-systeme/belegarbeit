%% The first command in your LaTeX source must be the \documentclass command.
\documentclass[acmtog]{acmart}
\usepackage[ngerman]{babel}
\usepackage[utf8]{inputenc}

\usepackage{csquotes}
\usepackage{listings}

\lstset{
    language=Python,
    basicstyle=\ttfamily\footnotesize,
    keywordstyle=\color{blue}\bfseries,
    stringstyle=\color{gray},
    commentstyle=\color{darkgray}\itshape,
    showstringspaces=false,
    numbers=left,
    numberstyle=\tiny\color{gray},
    breaklines=true,
    frame=single,
    captionpos=b,
}

\usepackage{tikz}
\usetikzlibrary{shapes,arrows,external}

%% \BibTeX command to typeset BibTeX logo in the docs
\AtBeginDocument{%
  \providecommand\BibTeX{{%
    \normalfont B\kern-0.5em{\scshape i\kern-0.25em b}\kern-0.8em\TeX}}}

\copyrightyear{2025}
\acmYear{2025}
\citestyle{acmauthoryear}

%\usepackage[textsize=tiny]{todonotes}

\usepackage[figurename=Fig.]{caption}
\setcopyright{none}
\makeatletter
\renewcommand{\fnum@figure}{Abb. \thefigure}
\makeatother
\addto\captionsngerman{\renewcommand{\figurename}{Abb.}}
\settopmatter{printacmref=false} % Removes citation information below abstract
\renewcommand\footnotetextcopyrightpermission[1]{} % removes footnote with conference information in first column

%%
%% end of the preamble, start of the body of the document source.
\begin{document}

\begin{CCSXML}
<ccs2012>
   <concept>
       <concept_id>10010405.10010406.10010417</concept_id>
       <concept_desc>Applied computing~Enterprise architectures</concept_desc>
       <concept_significance>500</concept_significance>
       </concept>
   <concept>
       <concept_id>10010583.10010750.10010751.10010757</concept_id>
       <concept_desc>Hardware~System-level fault tolerance</concept_desc>
       <concept_significance>300</concept_significance>
       </concept>
   <concept>
       <concept_id>10010520.10010575</concept_id>
       <concept_desc>Computer systems organization~Dependable and fault-tolerant systems and networks</concept_desc>
       <concept_significance>500</concept_significance>
       </concept>
 </ccs2012>
\end{CCSXML}

\ccsdesc[500]{Applied computing~Enterprise architectures}
\ccsdesc[300]{Hardware~System-level fault tolerance}
\ccsdesc[500]{Computer systems organization~Dependable and fault-tolerant systems and networks}

\keywords{Distributed Computing, High Availability, Resilience}
%%
%% The "title" command has an optional parameter,
%% allowing the author to define a "short title" to be used in page headers.
\title{Resilienz und Fehlertoleranz in verteilten Systemen}

%%
%% The "author" command and its associated commands are used to define
%% the authors and their affiliations.
%% Of note is the shared affiliation of the first two authors, and the
%% "authornote" and "authornotemark" commands
%% used to denote shared contribution to the research.
\author{Maksym Derhachov}
\authornote{Alle Studierenden trugen zu gleichen Teilen zu dieser Arbeit bei.}
\author{Florian Schmidt}
\authornotemark[1]
\author{Lukas Westholt}
\authornotemark[1]
\affiliation{%
  \institution{Hochschule für Technik, Wirtschaft und Kultur Leipzig (HTWK Leipzig)}
  \streetaddress{Karl-Liebknecht-Str. 132}
  \city{Leipzig}
  %\state{Ohio}
  \country{Deutschland}
  \postcode{04277}
}
%%
%% By default, the full list of authors will be used in the page
%% headers. Often, this list is too long, and will overlap
%% other information printed in the page headers. This command allows
%% the author to define a more concise list
%% of authors' names for this purpose.
\renewcommand{\shortauthors}{Derhachov, Schmidt und Westholt}

%%
%% The abstract is a short summary of the work to be presented in the
%% article.
\begin{abstract}
%  (Abstract-Länge ist typischerweise 15-25 Zeilen lang, in der PDF-Darstellung)

Diese Arbeit untersucht Strategien und Muster zur Steigerung der Resilienz moderner verteilter Anwendungen,
darunter Circuit Breaker, Retry-Muster und Fallback-Strategien.
Ziel ist es, Systeme gegen Ausfälle und Spitzenbelastungen abzusichern, betriebliche Kosten zu reduzieren
und Nutzerzufriedenheit zu gewährleisten.
Die Analyse umfasst allgemeine Ansätze wie Redundanz und Skalierung sowie konkrete Implementierungen in Python
und Beispiele aus der Industrie.
Abschließend werden zentrale Erkenntnisse zusammengefasst und Handlungsempfehlungen abgeleitet.
\end{abstract}

\maketitle

\section{Einleitung und Motivation}

\subsection*{Einleitung}


Die rasante technologische Entwicklung im Bereich verteilter Systeme und Cloud-basierter Anwendungen stellt
die Softwarearchitektur vor neue Herausforderungen.
Eine zentrale Motivation für die vorliegende Arbeit ist die Notwendigkeit, Anwendungen effizient gegen
Ausfälle und Störungen abzusichern, um langfristig betriebliche Kosten zu reduzieren und die Nutzerzufriedenheit
zu gewährleisten.
Gerade in geschäftskritischen Anwendungen können Systemausfälle erhebliche wirtschaftliche Verluste und
Vertrauensverlust bei den Endbenutzern verursachen.
Darüber hinaus sind in regulierten Branchen, wie der Finanz- und Gesundheitsindustrie,
hohe Anforderungen an Verfügbarkeit und Stabilität zu erfüllen.
Diese regulatorischen Vorgaben machen es notwendig, Resilienzstrategien nicht nur als Zusatz,
sondern als integralen Bestandteil der Softwareentwicklung zu betrachten.

Ein weiteres Schlüsselmotiv ist die Weiterentwicklung technologischer Ansätze zur Resilienzsteigerung.
Mit der zunehmenden Verbreitung von Microservices, Containerisierung und cloud-nativen Architekturen
entstehen neue Potenziale, aber auch Herausforderungen, die innovative Muster und Techniken erfordern.
Techniken wie Circuit Breaker, Retry-Muster und Fallback-Strategien sind hier entscheidend, um Fehler zu isolieren,
die Stabilität des Gesamtsystems zu erhalten und negative Auswirkungen auf Benutzer zu minimieren.
Denn Anwendungen müssen nicht nur unter normalen Betriebsbedingungen eine hohe Leistung erbringen,
sondern auch unter Spitzenbelastungen und bei unerwarteten Systemausfällen robuste Funktionalität gewährleisten.


Die iterative Integration solcher Resilienzstrategien in den Entwicklungsprozess wird durch agile Methoden begünstigt.
Agile Ansätze ermöglichen eine schrittweise Identifikation, Implementierung und Evaluierung von Resilienzanforderungen.
Dadurch wird sichergestellt, dass Systeme nicht nur initial robust gestaltet werden,
sondern sich kontinuierlich an neue Anforderungen und Bedrohungen anpassen können.


Die vorliegende Arbeit untersucht umfassend Strategien und Muster zur Steigerung der Resilienz
und Fehlertoleranz moderner webbasierter Anwendungen.
Ziel ist es, aktuelle Architekturansätze und Mechanismen systematisch einzuordnen und ihre Effektivität sowie
Einsatzpotenziale zu bewerten.
Dazu werden zunächst allgemeine Resilienzstrategien wie Redundanz, Partitionierung und Skalierung vorgestellt (Kapitel 2).
Im weiteren Verlauf werden spezifische Patterns wie Circuit Breaker, Retry-Muster und Fallback-Strategien detailliert
analysiert und hinsichtlich ihrer Vor- und Nachteile sowie Erfolgsfaktoren untersucht (Kapitel 3).


Der praktische Teil der Arbeit umfasst neben der Diskussion von Anwendungsbeispielen aus der Industrie auch
Beispielimplementierungen in der Programmiersprache Python (Kapitel 4).
Als Fallstudie dient Netflix, ein Vorreiter in der Entwicklung und Implementierung resilienter Architekturen.
Ziel ist es, übertragbare Handlungsempfehlungen und Entscheidungshilfen für die Auswahl geeigneter Resilienzmaßnahmen
abzuleiten.
Dabei werden sowohl die Grenzen der untersuchten Strategien als auch ihre potenziellen Risiken beleuchtet,
um ein umfassendes Verständnis zu gewährleisten.


Durch diese Arbeit wird ein Beitrag zur Weiterentwicklung robuster, skalierbarer und fehlertoleranter Systeme geleistet,
die den hohen Anforderungen moderner Webanwendungen gerecht werden.
Abschließend werden die zentralen Erkenntnisse zusammengefasst (Kapitel 5) und ein Ausblick auf zukünftige
Forschungsschwerpunkte im Bereich der Resilienz gegeben.

\section{Resilienz- und Fehlertoleranzstrategien}

Resilienz ist definiert als die Fähigkeit, Teilausfälle zu bewältigen und die Ausführung ohne Absturz fortzusetzen.

(Begriffe und Definitionen)

\section{(Hauptteil mit ggf. mehreren Sections)}

(der Hauptteil umfasst typischerweise ca. 2/3 bis 3/4 des Texts der Arbeit.)

Modifying the template --- including but not limited to: adjusting
margins, typeface sizes, line spacing, paragraph and list definitions,
and the use of the \verb|\vspace| command to manually adjust the
vertical spacing between elements of your work --- is not allowed.

{\bfseries Your document will be returned to you for revision if
  modifications are discovered.}

The ``\verb|acmart|'' document class requires the use of the
``Libertine'' typeface family. Your \TeX\ installation should include
this set of packages. Please do not substitute other typefaces. The
``\verb|lmodern|'' and ``\verb|ltimes|'' packages should not be used,
as they will override the built-in typeface families.

The ``\verb|acmart|'' document class includes the ``\verb|booktabs|''
package --- \url{https://ctan.org/pkg/booktabs} --- for preparing
high-quality tables.

Table captions are placed {\itshape above} the table.

Because tables cannot be split across pages, the best placement for
them is typically the top of the page nearest their initial cite.  To
ensure this proper ``floating'' placement of tables, use the
environment \textbf{table} to enclose the table's contents and the
table caption.  The contents of the table itself must go in the
\textbf{tabular} environment, to be aligned properly in rows and
columns, with the desired horizontal and vertical rules.  Again,
detailed instructions on \textbf{tabular} material are found in the
\textit{\LaTeX\ User's Guide}.

Immediately following this sentence is the point at which
Table~\ref{tab:freq} is included in the input file; compare the
placement of the table here with the table in the printed output of
this document.

\begin{table}
  \caption{Frequency of Special Characters}
  \label{tab:freq}
  \begin{tabular}{ccl}
    \toprule
    Non-English or Math&Frequency&Comments\\
    \midrule
    \O & 1 in 1,000& For Swedish names\\
    $\pi$ & 1 in 5& Common in math\\
    \$ & 4 in 5 & Used in business\\
    $\Psi^2_1$ & 1 in 40,000& Unexplained usage\\
  \bottomrule
\end{tabular}
\end{table}

To set a wider table, which takes up the whole width of the page's
live area, use the environment \textbf{table*} to enclose the table's
contents and the table caption.  As with a single-column table, this
wide table will ``float'' to a location deemed more
desirable. Immediately following this sentence is the point at which
Table~\ref{tab:commands} is included in the input file; again, it is
instructive to compare the placement of the table here with the table
in the printed output of this document.

\begin{table*}
  \caption{Some Typical Commands (table with full page width)}
  \label{tab:commands}
  \begin{tabular}{ccl}
    \toprule
    Command &A Number & Comments\\
    \midrule
    \texttt{{\char'134}author} & 100& Author \\
    \texttt{{\char'134}table}& 300 & For tables\\
    \texttt{{\char'134}table*}& 400& For wider tables\\
    \bottomrule
  \end{tabular}
\end{table*}

Always use midrule to separate table header rows from data rows, and
use it only for this purpose. This enables assistive technologies to
recognise table headers and support their users in navigating tables
more easily.

\subsection{Math Equations}
You may want to display math equations in three distinct styles:
inline, numbered or non-numbered display.  Each of the three are
discussed in the next sections.

\subsubsection{Inline (In-text) Equations}
A formula that appears in the running text is called an inline or
in-text formula.  It is produced by the \textbf{math} environment,
which can be invoked with the usual
\texttt{{\char'134}begin\,\ldots{\char'134}end} construction or with
the short form \texttt{\$\,\ldots\$}. You can use any of the symbols
and structures, from $\alpha$ to $\omega$, available in
\LaTeX~\cite{Lamport:LaTeX}; this section will simply show a few
examples of in-text equations in context. Notice how this equation:
\begin{math}
  \lim_{n\rightarrow \infty}x=0
\end{math},
set here in in-line math style, looks slightly different when
set in display style.  (See next section).

\subsubsection{Display Equations}
A numbered display equation---one set off by vertical space from the
text and centered horizontally---is produced by the \textbf{equation}
environment. An unnumbered display equation is produced by the
\textbf{displaymath} environment.

Again, in either environment, you can use any of the symbols and
structures available in \LaTeX\@; this section will just give a couple
of examples of display equations in context.  First, consider the
equation, shown as an inline equation above:
\begin{equation}
  \lim_{n\rightarrow \infty}x=0
\end{equation}
Notice how it is formatted somewhat differently in
the \textbf{displaymath}
environment.  Now, we'll enter an unnumbered equation:
\begin{displaymath}
  \sum_{i=0}^{\infty} x + 1
\end{displaymath}
and follow it with another numbered equation:
\begin{equation}
  \sum_{i=0}^{\infty}x_i=\int_{0}^{\pi+2} f
\end{equation}
just to demonstrate \LaTeX's able handling of numbering.

The ``\verb|figure|'' environment should be used for figures. One or
more images can be placed within a figure. If your figure contains
third-party material, you must clearly identify it as such, as shown
in the example below.
\begin{figure}[t]
  \centering
  \includegraphics[width=\linewidth]{images/htwkleipzig_sailing_frigate_full_set_sails_calm_ocean_blue_sky__4060fe96-060b-4ddf-890a-349f3de73f6f.png}
  \caption{Sailing Frigate. Image generated by Andreas Both (2023).}
    \label{fig:my-figure-1}
  \Description{A woman and a girl in white dresses sit in an open car.}
\end{figure}

\begin{figure*}[t]
  \centering
  \includegraphics[width=\linewidth]{images/anbo_de_A_group_of_Meerkats_in_nature_working_on_laptops_super__fd9893f8-7134-4950-94ae-0a24ff6ae5f3.png}
  \caption{Meerkats in nature working on laptops. Image by Andreas Both (2023). (image with full page width)}
    \label{fig:my-figure-2}
  \Description{A woman and a girl in white dresses sit in an open car.}
\end{figure*}

Your figures (cf. Abb.~\ref{fig:my-figure-1} and \ref{fig:my-figure-2}) should contain a caption which describes the figure to
the reader.

Figure captions are placed {\itshape below} the figure.

Every figure should also have a figure description unless it is purely
decorative. These descriptions convey what’s in the image to someone
who cannot see it. They are also used by search engine crawlers for
indexing images, and when images cannot be loaded.

A figure description must be unformatted plain text less than 2000
characters long (including spaces).  {\bfseries Figure descriptions
  should not repeat the figure caption – their purpose is to capture
  important information that is not already provided in the caption or
  the main text of the paper.} For figures that convey important and
complex new information, a short text description may not be
adequate. More complex alternative descriptions can be placed in an
appendix and referenced in a short figure description. For example,
provide a data table capturing the information in a bar chart, or a
structured list representing a graph.  For additional information
regarding how best to write figure descriptions and why doing this is
so important, please see
\url{https://www.acm.org/publications/taps/describing-figures/}.

\subsection{The ``Teaser Figure''}

A ``teaser figure'' is an image, or set of images in one figure, that
are placed after all author and affiliation information, and before
the body of the article, spanning the page. If you wish to have such a
figure in your article, place the command immediately before the
\verb|\maketitle| command:
\begin{verbatim}
  \begin{teaserfigure}
    \includegraphics[width=\textwidth]{sampleteaser}
    \caption{figure caption}
    \Description{figure description}
  \end{teaserfigure}
\end{verbatim}

\subsection{Citations and Bibliographies}

Citations and references are numbered by default. A small number of
ACM publications have citations and references formatted in the
``author year'' style; for these exceptions, please include this
command in the {\bfseries preamble} (before the command
``\verb|\begin{document}|'') of your \LaTeX\ source:
\begin{verbatim}
  \citestyle{acmauthoryear}
\end{verbatim}
\section{Diskussion}

(Einordnung, Interpretation und Bewertung der Erkenntnisse -- (nachvollziehbare, begründbare) Meinungen sind erlaubt)

\section{Zusammenfassung und Ausblick}

(Überblick über die gesamte Arbeit, Rückführung auf Aussagen aus Kapitel 1 durchführen, offene Punkte als neue Forschungsfragen definieren)

Papers may be written in languages other than English or include
titles, subtitles, keywords and abstracts in different languages (as a
rule, a paper in a language other than English should include an
English title and an English abstract).  Use \verb|language=...| for
every language used in the paper.  The last language indicated is the
main language of the paper.  

The title, subtitle, keywords and abstract will be typeset in the main
language of the paper.  The commands \verb|\translatedXXX|, \verb|XXX|
begin title, subtitle and keywords, can be used to set these elements
in the other languages.  The environment \verb|translatedabstract| is
used to set the translation of the abstract.  These commands and
environment have a mandatory first argument: the language of the
second argument.  See \verb|sample-sigconf-i13n.tex| file for examples
of their usage.


%% The next two lines define the bibliography style to be used, and
%% the bibliography file.
\bibliographystyle{ACM-Reference-Format}
\bibliography{../belegarbeit-literatur}

% \input{sections/anhang}

\end{document}
\endinput
%%
%% End of file `sample-acmtog.tex'.
