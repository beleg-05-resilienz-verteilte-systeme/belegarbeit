\subsection{Failover-Muster}
Eine weitere wichtige Methode, die sich in diesem Kontext bewährt hat, ist das
Failover-Muster. Dieses Konzept ist darauf ausgerichtet, die Kontinuität eines
Systems auch bei Störungen oder Ausfällen sicherzustellen, indem es eine
nahtlose Umschaltung auf alternative Ressourcen oder Backup-Komponenten
ermöglicht. Solche Mechanismen sind besonders in verteilten Systemen und bei
Diensten, die auf hohe Verfügbarkeit und Zuverlässigkeit angewiesen sind,
unverzichtbar. Das Failover-Muster implementiert eine Architektur, die in der Lage
ist, bei einem Ausfall der primären Komponente automatisch eine redundante
Komponente zu aktivieren, ohne dass die Endbenutzer von der Störung betroffen
sind.

Es gibt verschiedene Arten von Failover, die sich in ihrer Architektur und ihrem
Einsatzzweck unterscheiden. Das Active-Passive-Failover, bei dem eine aktive
Hauptkomponente von einer passiven Backup-Komponente unterstützt wird, wird
häufig für kosteneffiziente Implementierungen genutzt. Das Active-Active-Failover
hingegen betreibt mehrere aktive Komponenten gleichzeitig, was eine bessere
Lastverteilung und kürzere Umschaltzeiten ermöglicht. Darüber hinaus gibt es
Hot, Warm und Cold Failover. Während Hot Failover eine sofortige Umschaltung
ermöglicht, indem die Backup-Komponente ständig einsatzbereit ist, bieten Warm
und Cold Failover kostengünstigere Alternativen mit längeren Umschaltzeiten, da
die Backup-Komponenten nur teilweise oder gar nicht vorab initialisiert sind.

Zu den Vorteilen des Failover-Musters gehört vor allem die Erhöhung der
Systemverfügbarkeit. Durch die Fähigkeit, automatisch auf redundante
Komponenten umzuschalten, können Ausfallzeiten minimiert werden, was die
Benutzererfahrung erheblich verbessert. Gleichzeitig stärkt das Muster die
Resilienz eines Systems, da es auf potenzielle Ausfälle vorbereitet ist und deren
Auswirkungen abfedert. Es bietet außerdem Flexibilität, da es in verschiedenen Systemtypen eingesetzt werden kann, sowohl in stateless als auch in stateful
Anwendungen, wobei es sich an die spezifischen Anforderungen anpassen lässt.

Dennoch bringt das Failover-Muster auch einige Nachteile mit sich. Die Implementierung erfordert zusätzliche Ressourcen wie Hardware oder virtuelle
Maschinen, was die Kosten erhöht. Insbesondere bei Active-Active-Setups kann
die Synchronisation von Daten und Zuständen die Systemleistung belasten und
zusätzliche Komplexität verursachen. Zudem können unerwartete Fehler im
Failover-Prozess selbst auftreten, die die Ausfallzeit verlängern oder neue
Probleme verursachen. Daher ist es entscheidend, Failover-Mechanismen
regelmäßig zu testen und zu optimieren, um ihre Zuverlässigkeit sicherzustellen \cite{failover-patterns}.