\section{Einleitung und Motivation}

Die rasante technologische Entwicklung im Bereich verteilter Systeme und Cloud-basierter Anwendungen stellt
die Softwarearchitektur vor neue Herausforderungen.
Eine zentrale Motivation für die vorliegende Arbeit ist die Notwendigkeit, Anwendungen effizient gegen
Ausfälle und Störungen abzusichern, um langfristig betriebliche Kosten zu reduzieren und die Nutzerzufriedenheit
zu gewährleisten.
Gerade in geschäftskritischen Anwendungen können Systemausfälle erheblichen wirtschaftlichen Verlust und
Vertrauensverlust bei den Nutzerinnen und Nutzern verursachen.
Darüber hinaus sind in regulierten Branchen, wie der Finanz- und Gesundheitsindustrie,
hohe Anforderungen an Verfügbarkeit und Stabilität zu erfüllen.
Diese regulatorischen Vorgaben machen es notwendig, Resilienzstrategien nicht nur als Zusatz,
sondern als integralen Bestandteil der Softwareentwicklung zu betrachten.

Ein weiteres Schlüsselmotiv ist die Weiterentwicklung technologischer Ansätze zur Resilienzsteigerung.
Mit der zunehmenden Verbreitung von Microservices, Containerisierung und cloud-nativen Architekturen
entstehen neue Potenziale, aber auch Herausforderungen, die innovative Muster und Techniken erfordern.
Techniken wie Circuit Breaker, Retry-Muster und Fallback-Strategien sind hier entscheidend, um Fehler zu isolieren,
die Stabilität des Gesamtsystems zu erhalten und negative Auswirkungen auf Benutzer zu minimieren.
Denn Anwendungen müssen nicht nur unter normalen Betriebsbedingungen eine hohe Leistung erbringen,
sondern auch unter Spitzenbelastungen und bei unerwarteten Systemausfällen robuste Funktionalität gewährleisten.


Die iterative Integration solcher Resilienzstrategien in den Entwicklungsprozess wird durch agile Methoden begünstigt.
Agile Ansätze ermöglichen eine schrittweise Identifikation, Implementierung und Evaluierung von Resilienzanforderungen,
wodurch sichergestellt wird, dass Systeme nicht nur initial robust gestaltet werden,
sondern sich kontinuierlich an neue Anforderungen und Bedrohungen anpassen können.


Die vorliegende Arbeit untersucht Strategien und Muster zur Steigerung der Resilienz
und Fehlertoleranz moderner webbasierter Anwendungen.
Ziel ist es, aktuelle Architekturansätze und Mechanismen systematisch einzuordnen und ihre Effektivität sowie
Einsatzpotenziale zu bewerten.
Dazu werden zunächst allgemeine Resilienzstrategien wie Redundanz, Partitionierung und Skalierung vorgestellt (Kapitel 2).
Im weiteren Verlauf werden spezifische Patterns detailliert analysiert
und hinsichtlich ihrer Vor- und Nachteile sowie Erfolgsfaktoren untersucht (Kapitel 3).


Der praktische Teil der Arbeit umfasst neben der Diskussion von Anwendungsbeispielen aus der Industrie auch
Beispielimplementierungen in der Scriptsprache Python (Kapitel 4).
Als Fallstudie dient Netflix, ein Vorreiter in der Entwicklung und Implementierung resilienter Architekturen. % Eventuell...
Abschließend werden die zentralen Erkenntnisse zusammengefasst (Kapitel 5).
Dabei ist es vorrangig, übertragbare Handlungsempfehlungen und Entscheidungshilfen
für die Auswahl geeigneter Resilienzmaßnahmen abzuleiten.

Durch diese Arbeit wird ein Beitrag zur Weiterentwicklung robuster, skalierbarer und fehlertoleranter Systeme geleistet,
die den hohen Anforderungen moderner Webanwendungen gerecht werden.
